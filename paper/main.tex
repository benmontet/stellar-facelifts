%% rnaastex.cls is the classfile used for Research Notes. It is derived
%% from aastex61.cls with a few tweaks to allow for the unique format required.
\documentclass[twocolumn]{aastex62}

%% Define new commands here
\newcommand\latex{La\TeX}

\newcommand{\tess}{{\it TESS}}
\newcommand{\kep}{{\it Kepler}}
\newcommand{\wf}{{\it WFIRST}}

\newcommand{\todo}[3]{{\color{#2} \emph{#1} TO DO: #3}}
\newcommand{\mebtodo}[1]{\todo{MEGAN}{cyan}{#1}}
\newcommand{\btmtodo}[1]{\todo{BEN}{red}{#1}}
\newcommand{\anytodo}[1]{\todo{ANYONE}{green}{#1}}

\usepackage{hyperref}

\newcommand{\chicago}{Department of Astronomy and Astrophysics, University of
Chicago, 5640 S. Ellis Ave, Chicago, IL 60637, USA}
\newcommand{\sagan}{Sagan Fellow}

\newcommand{\flatiron}{Flatiron Institute, Simons Foundation, 162 Fifth Ave, New York, NY 10010, USA}

\shorttitle{Finding Hot Jupiters Working Title}
\shortauthors{Montet and Bedell}


\begin{document}

\title{Detecting Hot Jupiters through Asynchronous Rotation}


%% Note that the corresponding author command and emails has to come
%% before everything else. Also place all the emails in the \email
%% command instead of using multiple \email calls.
\correspondingauthor{Benjamin T. Montet}
\email{bmontet@uchicago.edu}



\author[0000-0001-7516-8308]{Benjamin~T.~Montet}
\altaffiliation{\sagan}
\affiliation{\chicago}

\author{Megan E. Bedell}
\affiliation{\flatiron}



%% The \author command can take an optional ORCID.


%% Note that RNAAS manuscripts DO NOT have abstracts.
%% See the online documentation for the full list of available subject
%% keywords and the rules for their use.

\begin{abstract}
    Working abstract here, we will fill this in later
\end{abstract}

\keywords{test
}

\section{Introduction}

Background here:

co-moving stars as wide binaries

wide binary components should be the same in many properties (esp. age), if they're not that's interesting: Poppenhaeger and Wolk 2014

gyrochronology: rotation + age link

rotation can be affected by HJs. we will look for this.



%\section{Asynchronous Rotation}


%Let's talk here about known HJs in wide binaries, and how many of them have faster rotation than their neighbors. We can use this to find HJs

%Show examples, if we can find one or more HJs with a wide binary companion in Kepler/K2.
%Do they also show excess flare activity?

%Let's also look at the percentage of hot jupiters with measured Prot in Kepler/K2

%This distribution is different than the field Prot observations? For what size planets/what periods?


\section{Known HJ hosts in Kepler and K2}
\subsection{Co-moving companions}

selection of HJ hosts

search for co-moving pairs

(FIGURE: error ellipses in parallax-PM space)

how many did we find?

(TABLE of pairs)

\begin{deluxetable*}{cccc}
\tablecaption{Likely co-moving pairs with Kepler HJ hosts.\label{tbl:kep-matches}}
\tablehead{\colhead{Primary KIC} & \colhead{Primary Gaia Source ID} & \colhead{Secondary Gaia Source ID} & \colhead{$\chi ^2$} }
\startdata
\enddata
\tablecomments{This table is published in its entirety in the machine-readable format. A portion is shown here for guidance regarding its form and content.}
\end{deluxetable*}

\begin{deluxetable*}{cccc}
\tablecaption{Likely co-moving pairs with K2 HJ hosts.\label{tbl:k2-matches}}
\tablehead{\colhead{Primary EPIC} & \colhead{Primary Gaia Source ID} & \colhead{Secondary Gaia Source ID} & \colhead{$\chi ^2$} }
\startdata
 201546283 &  3798552815560689664 &  3798552811267493888 & 5.18 \\
 201637175 &  3811002791880297472 &  3811002787586327040 & 2.10 \\
 201683540 &  3811900543124260352 &  3811900543123607552 & 0.80 \\
 202066192 &  3432091152007070720 &  3432092633770951680 & 9.78 \\
 202066192 &  3432091152007070720 &  3432091087582731776 & 6.87 \\
\enddata
\tablecomments{This table is published in its entirety in the machine-readable format. A portion is shown here for guidance regarding its form and content.}
\end{deluxetable*}


\subsection{Rotation periods of pairs}

how many of these have measurable rotation periods?

(FIGURE: multipanel example of autocorrelation functions)

are they significantly different?

(TABLE)

\subsection{Other disparate age indicators}

flaring?

Do we see any beaming signals in any of these stars? Do we xmatch with the Milholland results?

x-ray sources in GALEX?



\section{Predicting HJ hosts}
\subsection{Kepler and K2}
using rotation periods in co-moving pairs without known planets to predict spun-up hosts

quantify observations needed

this will allow us to find non-transiting HJs with known ages

\subsection{TESS}

What 2-minute cadence targets have a co-moving companion?

do some wild extrapolation to FFI stars%

\section{Conclusions}




\acknowledgments


Work by B.T.M. was performed under contract with the Jet Propulsion
Laboratory (JPL) funded by NASA through the Sagan Fellowship Program executed
by the NASA Exoplanet Science Institute.


\software{%
  numpy \citep{numpy}, 
  }
 
\facility{Kepler, Gaia}
 
\bibliography{papers}


\end{document}
%
