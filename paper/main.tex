%% rnaastex.cls is the classfile used for Research Notes. It is derived
%% from aastex61.cls with a few tweaks to allow for the unique format required.
\documentclass[twocolumn]{aastex62}

%% Define new commands here
\newcommand\latex{La\TeX}

\newcommand{\tess}{{\it TESS}}
\newcommand{\kep}{{\it Kepler}}
\newcommand{\wf}{{\it WFIRST}}

\newcommand{\todo}[3]{{\color{#2} \emph{#1} TO DO: #3}}
\newcommand{\mebtodo}[1]{\todo{MEGAN}{cyan}{#1}}
\newcommand{\btmtodo}[1]{\todo{BEN}{red}{#1}}
\newcommand{\anytodo}[1]{\todo{ANYONE}{green}{#1}}

\usepackage{hyperref}
\usepackage{amsmath}

\newcommand{\chicago}{Department of Astronomy and Astrophysics, University of
Chicago, 5640 S. Ellis Ave, Chicago, IL 60637, USA}
\newcommand{\sagan}{Sagan Fellow}

\newcommand{\flatiron}{Flatiron Institute, Simons Foundation, 162 Fifth Ave, New York, NY 10010, USA}

\shorttitle{Finding Hot Jupiters Working Title}
\shortauthors{Montet and Bedell}


\begin{document}

\title{Detecting Hot Jupiters through Asynchronous Rotation}


%% Note that the corresponding author command and emails has to come
%% before everything else. Also place all the emails in the \email
%% command instead of using multiple \email calls.
\correspondingauthor{Benjamin T. Montet}
\email{bmontet@uchicago.edu}



\author[0000-0001-7516-8308]{Benjamin~T.~Montet}
\altaffiliation{\sagan}
\affiliation{\chicago}

\author[0000-0001-9907-7742]{Megan E. Bedell}
\affiliation{\flatiron}



%% The \author command can take an optional ORCID.


%% Note that RNAAS manuscripts DO NOT have abstracts.
%% See the online documentation for the full list of available subject
%% keywords and the rules for their use.

\begin{abstract}
    Working abstract here, we will fill this in later.
    Words about tidal Q as well?
\end{abstract}

\keywords{test
}




\section{Introduction}

Background here:

co-moving stars as wide binaries

wide binary components should be the same in many properties (esp. age), if they're not that's interesting: Poppenhaeger and Wolk 2014

gyrochronology: rotation + age link

rotation can be affected by HJs. we will look for this.


\section{The Rotation Distribution of Hot Jupiters}

Let's talk about the distribution of rotation periods we measure for planet hosts.

We select all KOIs regardless of disposition based on transit depth, a/Rstar, period. Period does better than a/R star, why are some a/R stars very wrong? I believe this is because of the transit models they use, building eccentric models, but I need to verify this.

The rotation distribution we see looks different for the HJs for some definition of HJs. Very similar to the orbital period distribution, while smaller/more distant planets look like the field population (See also Ceillier15}.

Okay, so HJs orbit rapidly rotating stars! Does that mean they're orbiting rapidly rotating stars because they're young and they spiral in quickly, or because they're spinning up their host stars? We can use \textbf{wide binaries} to figure that out.



%\section{Asynchronous Rotation}


%Let's talk here about known HJs in wide binaries, and how many of them have faster rotation than their neighbors. We can use this to find HJs

%Show examples, if we can find one or more HJs with a wide binary companion in Kepler/K2.
%Do they also show excess flare activity?

%Let's also look at the percentage of hot jupiters with measured Prot in Kepler/K2

%This distribution is different than the field Prot observations? For what size planets/what periods?


\section{Known HJ hosts in Kepler and K2}
\subsection{Co-moving companions}

selection of HJ hosts: orbital period criteria, Rp/Rs criteria.

From this catalog of candidate host stars, we performed a cross-match to the Gaia DR2 database to find their astrometric properties. A match between the catalogs was considered valid if the distance between the Kepler source and the Gaia source was less than 0.75 arcseconds and the difference in Kepler K magnitude and Gaia G magnitude for the sources was less than 1.5 magnitudes. The distance between source coordinates used in this cut was calculated at the reference epoch J2000, meaning that the Gaia coordinates for each source were moved from the DR2 reported values at JD 2015.5 to their predicted values at JD 2000.0 using the Gaia proper motions. Of the 1699 candidate host stars, 1654 could be matched to a Gaia source. We speculate that the remaining 45 sources, many of which were known eclipsing binaries, may have been cut from the Gaia DR2 catalog due to their excess photometric variability (cite).

Given the Gaia DR2 sources corresponding to each Kepler candidate host star, we performed a Gaia database search for nearby sources within 1 arcminute that are co-moving with the Kepler star. We consider sources to be co-moving if the following requirements are satisfied:
\begin{itemize}
    \item both sources have parallax measured to SNR $\geq$ 5
    \item both sources have significantly non-zero proper motion (SNR $\geq$ 3 for $\mu_{RA}$ and/or $\mu_{Dec}$)
    \item both sources have low astrometric jitter as defined by \citet{GaiaHR}
    \item both sources have more than 8 visibility periods used in their astrometric solution
    \item the two sources' parallaxes and proper motions agree to $\chi^2 <= 5$, where $\chi_{m,n}^2$ for sources $m$ and $n$ is defined as:
    \begin{equation}
    \chi_{m,n}^2 = (X_m - X_n) (C_m + C_n)^{-1} (X_m - X_n) , 
    \end{equation}
    with $X_n$ as:
    \begin{equation}
    X_n = 
    \begin{bmatrix}
    \pi_n \\
    \mu_{RA, n} \\    
    \mu_{Dec, n}
    \end{bmatrix}
    \end{equation}
    and $C_n$ as the corresponding covariance matrix.
\end{itemize}


And then more here!

\subsection{Our favorite system}

KIC 6507427 is the bright star, KIC 6507433 is the faint star.

Ephemeris match! Morton+ show 20\% FPP for the bright one and 80\% for the faint one (why?)

PSF modeling, transit and rapid rotation clearly on the faint star

The older star has no periodicity, but can we measure an isochronal age?

Yes! With isochrones, it has to be evolved. Also it can't be two stars, still has to be evolved.

We can also use flicker, and this gives a logg of 2.70 to 2.90 depending on individual quarters. So we have an age, and it is old!

Bonus: can we say something about tidal Q since we know the age of the system?

Other bonus: Can we say anything about flare rates for this star? Or pull anything from the literature?


\section{EVERYTHING BELOW THIS IS OLD NOTES and POSSIBLY DIFFERENT THAN WHAT WE ACTUALLY DO}

(FIGURE: error ellipses in parallax-PM space)

how many did we find?

KOI 2455 == 6507427 (other is KOI-3815) 20\%, 80\% FPP from Morton+. We show transit is on the primary from PSF modeling (going to have to describe that). 

\begin{deluxetable*}{cccc}
\tablecaption{Likely co-moving pairs with Kepler HJ hosts.\label{tbl:kep-matches}}
\tablehead{\colhead{Primary KIC} & \colhead{Primary Gaia Source ID} & \colhead{Secondary Gaia Source ID} & \colhead{$\chi ^2$} }
\startdata
\enddata
\tablecomments{This table is published in its entirety in the machine-readable format. A portion is shown here for guidance regarding its form and content.}
\end{deluxetable*}

\begin{deluxetable*}{cccc}
\tablecaption{Likely co-moving pairs with K2 HJ hosts.\label{tbl:k2-matches}}
\tablehead{\colhead{Primary EPIC} & \colhead{Primary Gaia Source ID} & \colhead{Secondary Gaia Source ID} & \colhead{$\chi ^2$} }
\startdata
 201546283 &  3798552815560689664 &  3798552811267493888 & 5.18 \\
 201637175 &  3811002791880297472 &  3811002787586327040 & 2.10 \\
 201683540 &  3811900543124260352 &  3811900543123607552 & 0.80 \\
 202066192 &  3432091152007070720 &  3432092633770951680 & 9.78 \\
 202066192 &  3432091152007070720 &  3432091087582731776 & 6.87 \\
\enddata
\tablecomments{This table is published in its entirety in the machine-readable format. A portion is shown here for guidance regarding its form and content.}
\end{deluxetable*}


\subsection{Rotation periods of pairs}

how many of these have measurable rotation periods?

(FIGURE: multipanel example of autocorrelation functions)

are they significantly different?

(TABLE)

\subsection{Other disparate age indicators}

flaring?

Do we see any beaming signals in any of these stars? Do we xmatch with the Milholland results?

x-ray sources in GALEX?



\section{Predicting HJ hosts}
\subsection{Kepler and K2}

We find 1 in Kepler with a transiting HJ. Transits have a 5-10\% probability for HJs, so there are 10-20 HJs with comoving companions detectable with this method in the Kepler field alone.

using rotation periods in co-moving pairs without known planets to predict spun-up hosts

quantify observations needed

this will allow us to find non-transiting HJs with known ages

Can we flip around section 2, given a measurable rotation period of 3-5 days and the occurrence rate of hot Jupiters, what is the relative probability it is young vs the probability it is being spun up? Can we make a statistical argument based on spin-down relations and HJ occurrence rates, assuming everything else is equal?

\subsection{TESS}

What 2-minute cadence targets have a co-moving companion?

do some wild extrapolation to FFI stars%

\section{Conclusions}


epic 212628258: false positive

\acknowledgments


Work by B.T.M. was performed under contract with the Jet Propulsion
Laboratory (JPL) funded by NASA through the Sagan Fellowship Program executed
by the NASA Exoplanet Science Institute.


\software{%
  numpy \citep{numpy}, 
  }
 
\facility{Kepler, Gaia}
 
\bibliography{papers}


\end{document}
%
